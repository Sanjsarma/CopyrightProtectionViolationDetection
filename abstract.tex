\documentclass[10pt]{article}
\usepackage[a4paper,bindingoffset=0.2in,%
            left=1in,right=1in,top=1in,bottom=1in,%
            footskip=.25in]{geometry}
\title{Model Engineering College Ernakulam
\\Department of Computer Engineering
\\B. Tech. Computer Science \& Engineering 
\\CS 451 Project Preliminary
\\ Project Proposal
\\Copyright Protection and Violation Detection using Blockchain
}

\author{
MDL18CS027 11 Ardra Mohan
\\MDL18CS068 31 Malavika Rajesh Vikraman
\\MDL18CS104 51 Sandra Jacob
\\MDL18CS106 52 Sanjana S}

\begin{document}

\maketitle
{\bf Keywords:} Copyright protection, Copyright violation, Blockchain, Perceptual hashing algorithm, Digital fingerprint

\abstract{}
Copyright refers to the legal right of the owner of intellectual property. Copyright ownership gives authors exclusive right to use their work and share it with those who they deem fit. Copyright protects the right of the authors, that is, the owners of the intellectual properties. \\
\indent With the development of technology and growth of the internet, services and contents are being distributed online via delivery channels and sharing systems. Because these platforms are widely used, tracing violations of copyright and ensuring secure distribution of content has become a problem for content owners. Copyright infringement is the use or production of copyright-protected material without the permission of the copyright holder. It is a big challenge in today’s world. The aim is to design a system involving a Blockchain network to which images can be uploaded and checked for copyright protection and violation detection.
\section{Proposed System}
\subsection{Problem Statement}
Today, everyone relies on digital content, which results in a significant number of copyright violations happening on a daily basis. With the development of the Internet industry, the creation and realization of all kinds of digital copyright works are developing from traditional paper form to electronic form. In view of the current environment, digital rights protection and trading have not achieved synchronous development. There are many problems and loopholes, which lead to frequent infringement of digital copyright. Currently, there are no effective ways to protect the rights and interests of the originators, which seriously undermines their initiative to create. This project aims to address the serious problem of copyright violations of images and provide a means for owners to protect their work. 
\subsection{Proposed Solution}
The aim is to create a system for protecting copyrights and also for detecting violations of copyrights, using Blockchain technology. The system includes: 
\begin{itemize}
   \item \textbf{Uploading image:} Users are able to upload images using a client application.
   \item \textbf{Generating digital fingerprint:} Digital fingerprints of images can be generated using perceptual hashing algorithms, which can be stored in Blockchain network and can be used for comparison whenever a new image is to be uploaded. 
   \item \textbf{Copyright violation detection:} For deducing whether a copyright violation is likely to occur, the generated image’s digital fingerprint is compared against stored fingerprints. If similar fingerprints are present, then a copyright violation has taken place and the information of the user who has uploaded the image is passed to the owner.
\end{itemize}
\begin{thebibliography}{9}
\addcontentsline{toc}{chapter}{References}
\bibitem{} O. Evsutin, A. Melman and R. Meshcheryakov, \emph{Digital Steganography and Watermarking for \quad Digital Images: A Review of Current Research Directions}, IEEE Access, September 2020
\bibitem{} E. Bellini, Y. Iraqi and E. Damiani, \emph{Blockchain-Based Distributed Trust and Reputation Management Systems: A Survey}, IEEE Access, January 2020
\bibitem{} Zhao C, Liu M, Yang Y, Zhao F and Chen S., \emph{Toward A Blockchain Based Image Network Copyright Transaction Protection Approach}, International Conference on Security with Intelligent Computing and Big-data Services, April 2019
\bibitem{} Ms Supriya Maglekar and Dr. Dinesha H.A., \emph{Block Chain: An Innovative Research Area} FOURTH International Conference on Computing Communication Control and Automation(ICCUBEA), 2018

\end{thebibliography}

\hrule
\vspace{.2in}
\begin{flushleft}
Internal Guide:
\vspace{.3in}
\\Guide name
\\Asst. Professor
\\Department of Computer Engineering 
\\Model Engineering College 
\end{flushleft}

\end{document}
